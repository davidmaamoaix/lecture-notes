\documentclass[12pt]{article}

\usepackage{amsmath}
\usepackage{amssymb}
\usepackage{amsthm}
\usepackage{pgfplots}
\usepackage{mathtools}
\usepackage{booktabs}
\usepackage{indentfirst}

\usetikzlibrary{angles, quotes}

\pgfplotsset{compat=newest}

\title{Multivariable Calculus}
\author{Linxuan Ma}

\newcommand{\mo}[1]{\lvert #1 \rvert}
\newcommand{\mos}[1]{\lvert #1 \rvert^2}
\newcommand{\mov}[1]{\lvert \vec{#1} \rvert}
\newcommand{\RR}{\mathbb{R}}
\newcommand{\p}{\partial}
\newcommand{\iv}[1]{\langle #1 \rangle}

\theoremstyle{definition}
\newtheorem{defn}{Definition}[section]
\newtheorem{ex}{Exercise}

\begin{document}
	\maketitle
	
	\abstract{Multivariable calculus is a useful concept in mathematical analysis. This document focuses on the differential and integral for multivariable calculus. The following notes are taken from MIT OCW $18.02$ Multivariable Calculus.}
	
	\section{Vectors}
	
	A vector $\vec{v}$ is a mathematical structure that has represents a tuple of direction and magnitude; its elements correspond to the cartesian coordinates of the represented point. A vector $\vec{v} \in \RR^n$ represents a point in n-dimension.
	
	The length $\mov{v}$ of $\vec{v} \in \RR^n$ is a scalar that represents its magnitude: $$\mov{v} = \sqrt{\sum_{i=0}^{n}(\vec{v}_i)^2}$$
	
	Note that a vector does not necessarily have a starting and ending point;  it merely represents an offset in a direction.
	
	\subsection{Vector Arithmetic}
	
	Multiplying a vector by a scalar scales the individual elements by the scalar:
	\begin{equation*}
		\begin{bmatrix}
			a \\ b
		\end{bmatrix} * c = 
		\begin{bmatrix}
			ac \\ bc
		\end{bmatrix}
	\end{equation*}
	
	Addition of vectors is element-wise:
	\begin{equation*}
	\begin{bmatrix}
		a \\ b
	\end{bmatrix} +
	\begin{bmatrix}
		c \\ d
	\end{bmatrix} =
	\begin{bmatrix}
		a + c \\ b + d
	\end{bmatrix}
	\end{equation*}
	
	\subsection{Dot Product}
	
	\begin{defn}
		The dot product of $\vec{a}$ and $\vec{b}$ is the sum of their element-wise product: $$\vec{a} * \vec{b} = \sum_{i=0}^n \vec{a}_i \vec{b}_i$$
	\end{defn}
	
	Geometrically, the product of two vector is the product of their magnitude and the $\cos$ of their angle: $$\vec{a} * \vec{b} = \mo{\vec{a}} \mo{\vec{b}} \cos \theta$$
	
	Consider vector $\vec{a} * \vec{a}$, the resulting product should be   $\mov{a}^2$, as $\cos 0$ is $1$. Considering a triangle with angles $a$, $b$ and $c$ with angle $\theta$ opposing side $c$, the above can be deduced from the law of $\cos$: $$\mov{c}^2 = \mov{a}^2 + \mov{b}^2 - 2\mo{a}\mo{b}\cos \theta$$
	
	Since $\vec{c} = \vec{a} - \vec{b}$, it can be deduced that:
	\begin{align*}
		\mov{c}^2 = \vec{c} * \vec{c} &= (\vec{a} - \vec{b})*(\vec{a} - \vec{b}) \\
		&= \vec{a} * \vec{a} - \vec{a} * \vec{b} - \vec{b} * \vec{a} + \vec{b} * \vec{b}\\
		&= \mov{a}^2 + \mov{b}^2 - 2\vec{a}\vec{b}
	\end{align*}
	
	As shown in the last line of the transformation of $\mov{c}^2$, $-2\vec{a}\vec{b} = -2\mov{a}\mov{b} \cos \theta$.
	
	\subsection{Applications of Dot Product}
	
	Dot product can be used to compute lengths and angles. Consider a $\RR^3$ space with $P = (1, 0, 0)$, $Q = (0, 1, 0)$ and $R = (0, 0, 2)$, the angle $\angle RPQ$ can be found with dot product:
	\begin{align*}
		\overrightarrow{PR} * \overrightarrow{PQ} &= \mo{\overrightarrow{PR}} \mo{\overrightarrow{PQ}} \cos \theta \\
		\cos \theta &= \frac{\overrightarrow{PQ} * \overrightarrow{PR}}{\mo{\overrightarrow{PQ}}\mo{\overrightarrow{PR}}}
	\end{align*}
	
	By plugging in our example, we obtain:
	\begin{gather*}
		\cos \theta = \frac{1 + 0 + 0}{\sqrt{2} * \sqrt{5}} = \frac{1}{\sqrt{10}}
	\end{gather*}
	
	Another application of dot product is to determine the orthogonality of two vectors, i.e. when are two vectors $a$ and $b$ perpendicular. Note that the sign of a dot product denotes the directional relation of the two vectors $a$ and $b$ with angle $\theta$:
	\begin{itemize}
		\item $\vec{a} * \vec{b} > 0$ if $\theta < 90$
		\item $\vec{a} * \vec{b} = 0$ if $\theta = 90$
		\item $\vec{a} * \vec{b} > 0$ if $\theta > 90$
	\end{itemize}
	
	Consider the linear equation $x + 2y + 3z = 0$. The above equation can be written in the form of a dot product:
	\begin{gather*}
		\begin{bmatrix}
			1 \\ 2 \\ 3
		\end{bmatrix} *
		\begin{bmatrix}
			x \\ y \\ z
		\end{bmatrix} = 0
	\end{gather*}
	Therefore, the vector $\iv{x, y, z}$ is perpendicular to $\iv{1, 2, 3}$, making the former a plane, as $\cos \theta = 0$ where $\theta$ is the angle between the two vectors.
	
\end{document}
