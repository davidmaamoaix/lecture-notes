\documentclass[12pt]{article}

\usepackage{amsmath}
\usepackage{amssymb}
\usepackage{amsthm}
\usepackage{pgfplots}
\usepackage{mathtools}
\usepackage{booktabs}
\usepackage{indentfirst}

\usetikzlibrary{angles, quotes}

\pgfplotsset{compat=newest}

\title{Multivariable Calculus}
\author{Linxuan Ma}

\newcommand{\mo}[1]{\lvert #1 \rvert}
\newcommand{\mos}[1]{\lvert #1 \rvert^2}
\newcommand{\mov}[1]{\lvert \vec{#1} \rvert}
\newcommand{\RR}{\mathbb{R}}
\newcommand{\p}{\partial}
\newcommand{\iv}[1]{\langle #1 \rangle}
\newcommand{\adj}{\text{adj}}

\theoremstyle{definition}
\newtheorem{defn}{Definition}[section]
\newtheorem{ex}{Exercise}

\begin{document}
	\maketitle
	
	\abstract{Multivariable calculus is a useful concept in mathematical analysis. This document focuses on the differential and integral for multivariable calculus. The following notes are taken from MIT OCW $18.02$ Multivariable Calculus.}
	
	\section{Vectors}
	
	A vector $\vec{v}$ is a mathematical structure that has represents a tuple of direction and magnitude; its elements correspond to the cartesian coordinates of the represented point. A vector $\vec{v} \in \RR^n$ represents a point in n-dimension.
	
	The length $\mov{v}$ of $\vec{v} \in \RR^n$ is a scalar that represents its magnitude: $$\mov{v} = \sqrt{\sum_{i=0}^{n}(\vec{v}_i)^2}$$
	
	Note that a vector does not necessarily have a starting and ending point;  it merely represents an offset in a direction.
	
	\subsection{Vector Arithmetic}
	
	Multiplying a vector by a scalar scales the individual elements by the scalar:
	\begin{equation*}
		\begin{bmatrix}
			a \\ b
		\end{bmatrix} * c = 
		\begin{bmatrix}
			ac \\ bc
		\end{bmatrix}
	\end{equation*}
	
	Addition of vectors is element-wise:
	\begin{equation*}
	\begin{bmatrix}
		a \\ b
	\end{bmatrix} +
	\begin{bmatrix}
		c \\ d
	\end{bmatrix} =
	\begin{bmatrix}
		a + c \\ b + d
	\end{bmatrix}
	\end{equation*}
	
	\subsection{Dot Product}
	
	\begin{defn}
		The dot product of $\vec{a}$ and $\vec{b}$ is the sum of their element-wise product: $$\vec{a} * \vec{b} = \sum_{i=0}^n \vec{a}_i \vec{b}_i$$
	\end{defn}
	
	Geometrically, the product of two vector is the product of their magnitude and the $\cos$ of their angle: $$\vec{a} * \vec{b} = \mo{\vec{a}} \mo{\vec{b}} \cos \theta$$
	
	Consider vector $\vec{a} * \vec{a}$, the resulting product should be   $\mov{a}^2$, as $\cos 0$ is $1$. Considering a triangle with angles $a$, $b$ and $c$ with angle $\theta$ opposing side $c$, the above can be deduced from the law of $\cos$: $$\mov{c}^2 = \mov{a}^2 + \mov{b}^2 - 2\mo{a}\mo{b}\cos \theta$$
	
	Since $\vec{c} = \vec{a} - \vec{b}$, it can be deduced that:
	\begin{align*}
		\mov{c}^2 = \vec{c} * \vec{c} &= (\vec{a} - \vec{b})*(\vec{a} - \vec{b}) \\
		&= \vec{a} * \vec{a} - \vec{a} * \vec{b} - \vec{b} * \vec{a} + \vec{b} * \vec{b}\\
		&= \mov{a}^2 + \mov{b}^2 - 2\vec{a}\vec{b}
	\end{align*}
	
	As shown in the last line of the transformation of $\mov{c}^2$, $-2\vec{a}\vec{b} = -2\mov{a}\mov{b} \cos \theta$.
	
	\subsection{Applications of Dot Product}
	
	Dot product can be used to compute lengths and angles. Consider a $\RR^3$ space with $P = (1, 0, 0)$, $Q = (0, 1, 0)$ and $R = (0, 0, 2)$, the angle $\angle RPQ$ can be found with dot product:
	\begin{align*}
		\overrightarrow{PR} * \overrightarrow{PQ} &= \mo{\overrightarrow{PR}} \mo{\overrightarrow{PQ}} \cos \theta \\
		\cos \theta &= \frac{\overrightarrow{PQ} * \overrightarrow{PR}}{\mo{\overrightarrow{PQ}}\mo{\overrightarrow{PR}}}
	\end{align*}
	
	By plugging in our example, we obtain:
	\begin{gather*}
		\cos \theta = \frac{1 + 0 + 0}{\sqrt{2} * \sqrt{5}} = \frac{1}{\sqrt{10}}
	\end{gather*}
	
	Another application of dot product is to determine the orthogonality of two vectors, i.e. when are two vectors $a$ and $b$ perpendicular. Note that the sign of a dot product denotes the directional relation of the two vectors $a$ and $b$ with angle $\theta$:
	\begin{itemize}
		\item $\vec{a} * \vec{b} > 0$ if $\theta < 90$
		\item $\vec{a} * \vec{b} = 0$ if $\theta = 90$
		\item $\vec{a} * \vec{b} > 0$ if $\theta > 90$
	\end{itemize}
	
	Consider the linear equation $x + 2y + 3z = 0$. The above equation can be written in the form of a dot product:
	\begin{gather*}
		\begin{bmatrix}
			1 \\ 2 \\ 3
		\end{bmatrix} *
		\begin{bmatrix}
			x \\ y \\ z
		\end{bmatrix} = 0
	\end{gather*}
	Therefore, the vector $\iv{x, y, z}$ is perpendicular to $\iv{1, 2, 3}$, making the former a plane, as $\cos \theta = 0$ where $\theta$ is the angle between the two vectors.
	
	\section{Cross Product}
	
	To get the component $a_u$ of $\vec{a}$ along unit vector $\vec{u}$:
	\begin{align*}
		a_u &= \mov{a}\cos{\theta} \\
		&= \mov{a}\mov{u}\cos{\theta} \\
		&= \vec{a} * \vec{u}
	\end{align*}
	
	\subsection{Geometric Interpretation of Determinants}
	
	To get the area of triangle $A$ with two edges denoted as vector $\vec{a}$ and $\vec{b}$ originating from point $O$, the area of $A$ is:
	\begin{gather*}
		\text{Area}(A) = \frac{1}{2} \mov{a}\mov{b} \sin \theta
	\end{gather*}
	
	The above could be obtained with dot product considering the complement angle and $\vec{a'}$ as $\vec{a}$ after $90^\circ$ rotation. $\angle a'Oa = \theta'$:
	\begin{align*}
		\text{Area}(A) &= \frac{1}{2} \mov{a}\mov{b} \sin \theta \\
		&= \vec{a'} * \vec{b} \\
		&= \iv{-a_2, a_1} * \iv{b_1, b_2} \\
		&= a_1 b_2 - a_2 b_2
	\end{align*}
	
	Note that the result of the above transformation is equivalent to the determinant of the matrix:
	\begin{equation*}
	\begin{bmatrix}
		a_1 & a_2 \\ b_1 & b_2
	\end{bmatrix}
	\end{equation*}
	
	\subsection{Determinant in $R^3$}
	
	The determinant of matrix $A = \iv{\vec{A}, \vec{B}, \vec{C}}^\top$ is:
	\begin{align*}
		\mo{A} = a_1
		\begin{vmatrix}
			b_2 & b_3 \\ c_2 & c_3
		\end{vmatrix} - a_2
		\begin{vmatrix}
			b_1 & b_3 \\ c_1 & c_3
		\end{vmatrix} + a_3
		\begin{vmatrix} 
			b_1 & b_2 \\ c_1 & c_2
		\end{vmatrix} 
	\end{align*}
	
	The determinant of matrix $\iv{\vec{a}, \vec{b}, \vec{c}}$ is the volume of the parallelepiped formed by edges $\vec{a}$, $\vec{b}$ and $\vec{c}$.
	
	\subsection{Rotation of Vectors}
	
	Vector $\vec{v} = \iv{a, b}$ after $90^\circ$ rotation gives $\iv{-b, a}$. Similarly, a clockwise rotation of $\vec{v}$ in the clockwise direction gives $\iv{b, -a}$.
	
	\subsection{Cross Product's Geometric Interpretation}
	
	The cross product of $\vec{a}$ and $\vec{b}$ is a vector $\vec{a} \times \vec{b}$. The above cross product can be represented via the "determinant" of the matrix:
	\begin{equation*}
	\begin{bmatrix}
		i & j & k \\ a_1 & a_2 & a_3 \\ b_1 & b_2 & b_3
	\end{bmatrix}
	\end{equation*}
	
	The coefficients of $i$, $j$ and $k$ are the $x$, $y$ and $z$ component of the resulting vector respectively.
	
	Consider the parallelepiped $A$ formed by edges $\vec{a}$, $\vec{b}$ and $\vec{c}$. Its volume is the area of the base scaled by the height $h_A$. We obtain:
	\begin{align*}
		\text{Area}(A) &= h_A * \mo{\vec{b} \times \vec{c}} \\
		&= \left(\vec{a} * \frac{\vec{b} \times \vec{c}}{\mo{\vec{b} \times \vec{c}}}\right) * \mo{\vec{b} \times \vec{c}} \\
		&= \vec{a} * (\vec{b} \times \vec{c})
	\end{align*}
	
	The above equation can be expanded via the definition of cross product to display its equality with the determinant of $\iv{\vec{a}, \vec{b}, \vec{c}}^\top$.
	
	\subsubsection{Right Hand Rule}
	
	The right hand rule is stupid. It doesn't work on people who can't tell left from right, like me. Soooooo basically given the cross product $\vec{e} = \vec{a} \times \vec{b}$ where $\vec{e}$ is in the position of poking your eyes out, $\vec{a}$ and $\vec{b}$ are positioned in a counter-clockwise manner.
	
	\section{Matrices}
	
	First, trivially: $$\vec{a} \times \vec{b} = -\vec{b} \times \vec{a}$$
	
	\subsection{Application of Cross Product}
	
	Give $p_1$, $p_2$ and $p_3$ in $\RR^3$ aligned along a hyperplane, the equation for the hyperplane $P(x, y, z)$ denotes the condition for $x$, $y$ and $z$ of a new given point $p$ for it to be contained in the hyperplane.
	
	One approach is to consider if the vectors $\overrightarrow{p_1p_3}$, $\overrightarrow{p_1p_2}$ and $\overrightarrow{p_1p}$ are in the same plane. In order words: $$\det(\iv{\overrightarrow{p_1p}, \overrightarrow{p_1p_2}, \overrightarrow{p_1p_3}}) = 0$$.
	
	Another approach is to obtain the cross product of $\overrightarrow{p_1p_2}$ and $\overrightarrow{p_1p_3}$, and see if the direction of the resultant vector (normal vector) is perpendicular to $\overrightarrow{p_1p}$. In other words: $$\overrightarrow{p_1p} * (\overrightarrow{p_1p_2} \times \overrightarrow{p_1p_3}) =  0$$
	
	\subsection{Linear Relations}
	
	Matrices encapsulate linear transformations. This is useful in scenarios such as change of basis, etc.
	
	Consider a change of basis from $\iv{x_1, x_2, x_3}$ to $\iv{u_1, u_2, u_3}$:
	\begin{align*}
		u_1 &= 2x_1 + 3x_2 + 3x_3 \\
		u_2 &= 2x_1 + 4x_2 + 5x_3 \\
		u_3 &= x_1 + x_2 + 2x_3
	\end{align*}
	
	The above transformation can be represented via a matrix:
	\begin{equation*}
		\begin{bmatrix}
			2 & 3 & 3 \\ 2 & 4 & 5 \\ 1 & 1 & 2
		\end{bmatrix} * \begin{bmatrix}
			x_1 \\ x_2 \\ x_3
		\end{bmatrix} = \begin{bmatrix}
			u_1 \\ u_2 \\ u_3
		\end{bmatrix}
	\end{equation*}
	
	The product of two matrices $A * X$ is the combinations of dot products between the rows of $A$ and the columns of $X$. Therefore, for dimension $A \in \RR^{m*n}$ and $X \in \RR^{n*o}$, $A*X \in \RR^{m*o}$. Trivially, two matrices are only multiplicable if the width of $A$ equals the height of $X$ in $A*X$.
	
	\subsection{Intuition of Matrix Multiplication}
	
	The transformation $AB$ (as matrix product) represents applying transformation $B$ then applying transformation $A$:
	$$(AB)x = A(Bx)$$
	
	In addition, matrix multiplication are not commutative, namely:
	$$AB \neq BA$$
	
	In other words, matrices form a semigroup under multiplication from what we've been currently given (actually a monoid, but identities are covered yet in lecture 3).
	
	(As I'm writing the last sentence the prof literally started to talk about identity matrices\dots Screw it monoid it is.)
	
	An identity matrix $I$ is a matrix that does nothing; it does not impose any linear transformation, and therefore for any matrix $A$:
	\begin{align*}
		AI &= A \\
		IA &= A
	\end{align*}
	
	An identity matrix has $1$s on the diagonal, and $0$s elsewhere. An $\RR^3$ identity matrix:
	\begin{equation*}
		\begin{bmatrix}
			1 & 0 & 0 \\ 0 & 1 & 0 \\ 0 & 0 & 1
		\end{bmatrix}
	\end{equation*}
	
	\subsection{Linear Transformation}
	
	For the linear transformation of rotation by $90^\circ$ counter-clockwise, the corresponding matrix is:
	\begin{equation*}
		\begin{bmatrix}
			0 & -1 \\ 1 & 0
		\end{bmatrix}
	\end{equation*}
	
	The above matrix performs the transformation:
	\begin{equation*}
		\begin{bmatrix}
			x \\ y
		\end{bmatrix} \to \begin{bmatrix}
			-y \\ x
		\end{bmatrix}
	\end{equation*}
	
	\subsubsection{Inverting Matrices}
	
	Linear transformations can be inverted (unless in certain scenarios). Similarly, matrices can be inverted, and their inverse represents the inverted linear transformation. Due to this, it is trivial that:
	$$AA^{-1} = I$$
	
	With this property, the equation $AX = B$ (solving for $X$) can be rewritten as $X = A^{-1}B$ as $A^{-1}A = I$ on the LHS.
	
	Generally, the inverse of matrix $A$ is:
	\begin{gather*}
		A^{-1} = \frac{1}{\det(A)} \adj(A)
	\end{gather*}
	
	$\adj(A)$ denotes the adjoint of $A$. To obtain the adjoint matrix we first calculate the minors, which has the same dimension as $A$, and each entry is the determinant of $A$ after removing the row and column that the entry belongs to. Consider:
	\begin{equation*}
		A = \begin{bmatrix}
			2 & 3 & 3 \\ 2 & 4 & 5 \\ 1 & 1 & 2
		\end{bmatrix}
	\end{equation*}
	
	Following the above steps, we obtain:
	\begin{equation*}
		\begin{pmatrix}
			3 & -1 & -2 \\ 3 & 1 & -1 \\ 3 & 4 & 2
		\end{pmatrix}
	\end{equation*}
	
	The cofactors is identical to the minors except for certain inversion in sign (signs are flipped in a checkerboard pattern) ($-$ denotes a flip in sign):
	\begin{equation*}
		\begin{matrix}
			+ & - & + \\ - & + & - \\ + & - & +
		\end{matrix}
	\end{equation*}
	
	Lastly, the matrix is transposed to obtain the adjoint matrix. The adjoint matrix for the above $A$ is:
	\begin{equation*}
		\begin{pmatrix}
			3 & -3 & 3 \\ 1 & 1 & -4 \\ -2 & 1 & 2
		\end{pmatrix}
	\end{equation*}
	
	\section{Equations of Planes}
	
	Recall that an equation for a plane is of the form (where $a$, $b$, $c$ and $d$ are constants):
	\begin{gather*}
		ax + by + cz = d
	\end{gather*}
	
	To illustrate, consider the equation for a plane through the origin with normal vector $\vec{N} = \iv{1, 5, 10}$. Trivially, for any point $\vec{p}$ to be in the plane of interest, $\vec{N} * \vec{p} = 0$.
	
	Similarly, consider the equation for a plane through $P_0 = \iv{2, 1, -1}$ and normal vector $\vec{N} = \iv{1, 5, 10}$, $\overrightarrow{P_0P} * \vec{N} = 0$. In the above case:
	\begin{align*}
		& \overrightarrow{P_0P} * \vec{N} = 0 \\
		\iff & \iv{x - 2, y - 1, z + 1} * \iv{1, 5, 10} = 0 \\
		\iff & (x - 2) + 5(y - 1) + 10(z + 1) = 0 \\
		\iff & x + 5y + 10z = -3
	\end{align*}
	
	An easier method to obtain $-3$ is just to plug $P_0 = \iv{2, 1, -1}$ into $x + 5y + 10z$, as $P_0$ acts as the "origin" of the plane.
	
	To generalize, in equation $ax + by + cz = d$, $\vec{N} = \iv{a, b, c}$ is a normal vector to the plane described by the above equation.
	
	As an example, consider $\vec{v} = \iv{1, 2, -1}$ and the plane described by $x + y + 3z = 5$. One normal vector of the plane is $\vec{N} = \iv{1, 1, 3}$. It is obvious that $\vec{v}$ is not perpendicular to the plane. $\vec{v}$ is perpendicular to $\vec{N}$ since $\vec{N} * \vec{v} = 0$, thus $\vec{v}$ is parallel to the plane.
	
	\subsection{Linear Systems}
	
	A linear system is essentially a bunch of linear equations:
	\begin{align*}
		x + z &= 1 \\
		x + y &= 2 \\
		x + 2y + 3z &= 3
	\end{align*}
	
	To find the common solution to the above system is to find a point $p$ that is contained in all three planes described by the above equations.
	
	The solution is not as trivial when a plane contains the line formed by the other two planes, then there are infinitely many answers, as there are no constraints restricting the line. Similarly, if all three planes are parallel and overlapping, the result is a plane.
	
	There is no solution for a linear system if the planes never intersects.
	
	As a result of the above phenomenon, $AX = B \iff X = A^{-1}B$ might not be always viable depending on the invertibility of the matrix $A$. A matrix $A$ is only invertible if the determinant of it is not $0$, i.e. $\det(A) \neq 0$, as $$\det(A) = 0 \iff \text{one plane is parallel to the intersection line}$$.
	
	\subsection{Homogeneous Case}
	
	A homogeneous case of the linear is when $AX = 0$, i.e. each equation is of the form $ax + by + cz = 0$. An obvious solution is $\iv{0, 0, 0}$, referred to as the \emph{trivial solution}. In other words, all the planes passes through the origin, making the origin a solution to the system.
	
	If $\det{A} \neq 0$, then the inverse of $A$ can be used to solve for $X$. However, this is not very helpful:
	\begin{gather*}
		AX = 0 \iff X = A^{-1}0 \Rightarrow X = \iv{0, 0, 0}
	\end{gather*}
	
	In the case that $\det(A) = 0$, recall that the coefficients of an equation is a normal vector to the plane:
	\begin{align*}
		&\det(A) = 0 \\
		\iff &\det(\vec{N_1}, \vec{N_2}, \vec{N_3}) = 0 \\
		\iff &\text{$\vec{N_1}$, $\vec{N_2}$ and $\vec{N_3}$ are coplanar}
	\end{align*}
	
	Recall that $\det(\vec{N_1}, \vec{N_2}, \vec{N_3}) = 0$ means the volume of the parallelepiped formed by the three vectors is $0$.
	
	Therefore, the solution is the line through $0$ that is perpendicular to $\vec{N_1}$, $\vec{N_2}$ and $\vec{N_3}$ (can be obtained via a simple cross product of the normal vectors). The solution in this case is non-trivial.
	
	\subsection{General Case}
	
	For linear system $AX = B$, if $\det(A) \neq 0$, then a unique solution exists: $X = A^{-1}B$. If $\det(A) = 0$, then either no solutions or infinitely many solutions. We are not yet able to distinguish between the two cases with what the course has currently covered.
	
	
	\section{Equation of Lines}
	
	Deriving from the previous lecture, we can see a line as the intersection of 2 planes. A more convenient representation of a line is by considering it as a trajectory of a moving point. This is referred to as a \emph{parametric equation}.
	
	As an example, consider points $Q_0$ and $Q_1$:
	\begin{align*}
		Q_0 &= \iv{-1, 2, 2} \\
		Q_1 &= \iv{1, 3, -1}
	\end{align*}
	
	The line can be described as a trajectory (originating from $Q_0$) between the two points, namely:
	\begin{align*}
		Q(t) &= Q_0 + t * (Q_1 - Q_0) \\
		&= Q_0 + t * \iv{2, 1, -3}
	\end{align*}
	
	Treating $Q$ as $Q(t) = \iv{x(t), y(t), z(t)}$ and considering point $Q_0$, we obtain that:
	\begin{align*}
		x(t) + 1 &= 2t \\
		y(t) - 2 &= t \\
		z(t ) - 2 &= -3t
	\end{align*}
	
	Reorder and we obtain:
	\begin{align*}
		x(t) &= 2t - 1 \\
		y(t) &= t + 2\\
		z(t ) &= -3t + 2
	\end{align*}
	
	In other words, we've just shown that $Q(t) = Q_0 + t * \overrightarrow{Q_0Q_1}$
	
	\subsection{Applications}
	
	The equation of a line can be used to check if it intersects with a plane and where. Consider the plane $x + 2y + 4z = 7$, what orientation do $Q_0$ and $Q_1$ have in respect to the plane?
	
	By plugging $Q_0$ and $Q_1$ into the plane equation, we obtain:
	\begin{align*}
		x + 2y + 4z &= -1 + 2*2 + 4*2 = 11 > 7 \\
		x + 2y + 4z &= 1 + 2*3 + 4*(-1) = 3 < 7
	\end{align*}
	
	Thus, neither $Q_0$ or $Q_1$ lies in the plane. However, by observing how the result of plugging in the points lie on different subspaces (as denoted by $>7$ and $<7$), $Q_0$ and $Q_1$ are on different side of the plane.
	
	To obtain the intersection of the plane with the line, we plug in $t$ into the line equation:
	\begin{align*}
		&x(t) + 2y(t) + 4z(t) \\
		= &(-1 + 2t) + 2(2 + t) + 4(2 - 3t) \\
		= &-8t + 11
	\end{align*}
	
	This is compared to plane equation: the line intersects at the point where $-8t + 11 = 7$.
	
	The specific point can be obtained via:
	\begin{align*}
		Q(\frac{1}{2}) = \iv{0, \frac{5}{2}, \frac{1}{2}}
	\end{align*}
	
	\subsection{Parametric Equations}
	
	Parametric Equations are excellent at representing a curve/trajectory in space. In general, parametric equations can represent arbitrary motions in a given space.
	
	\begin{defn}
		A \emph{cycloid} is formed as the trajectory of a point on a rolling wheel with radius $r$.
	\end{defn}
	
\end{document}
