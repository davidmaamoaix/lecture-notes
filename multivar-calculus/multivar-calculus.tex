\documentclass[12pt]{article}

\usepackage{amsmath}
\usepackage{amssymb}
\usepackage{amsthm}
\usepackage{pgfplots}
\usepackage{mathtools}
\usepackage{booktabs}
\usepackage{indentfirst}

\usetikzlibrary{angles, quotes}

\pgfplotsset{compat=newest}

\title{Multivariable Calculus}
\author{Linxuan Ma}

\newcommand{\mo}[1]{\lvert #1 \rvert}
\newcommand{\mos}[1]{\lvert #1 \rvert^2}
\newcommand{\mov}[1]{\lvert \vec{#1} \rvert}
\newcommand{\RR}{\mathbb{R}}
\newcommand{\p}{\partial}
\newcommand{\iv}[1]{\langle #1 \rangle}

\theoremstyle{definition}
\newtheorem{defn}{Definition}[section]
\newtheorem{ex}{Exercise}

\begin{document}
	\maketitle
	
	\abstract{Multivariable calculus is a useful concept in mathematical analysis. This document focuses on the differential and integral for multivariable calculus. The following notes are taken from MIT OCW $18.02$ Multivariable Calculus.}
	
	\section{Vectors}
	
	A vector $\vec{v}$ is a mathematical structure that has represents a tuple of direction and magnitude; its elements correspond to the cartesian coordinates of the represented point. A vector $\vec{v} \in \RR^n$ represents a point in n-dimension.
	
	The length $\mov{v}$ of $\vec{v} \in \RR^n$ is a scalar that represents its magnitude: $$\mov{v} = \sqrt{\sum_{i=0}^{n}(\vec{v}_i)^2}$$
	
	Note that a vector does not necessarily have a starting and ending point;  it merely represents an offset in a direction.
	
	\subsection{Vector Arithmetic}
	
	Multiplying a vector by a scalar scales the individual elements by the scalar:
	\begin{equation*}
		\begin{bmatrix}
			a \\ b
		\end{bmatrix} * c = 
		\begin{bmatrix}
			ac \\ bc
		\end{bmatrix}
	\end{equation*}
	
	Addition of vectors is element-wise:
	\begin{equation*}
	\begin{bmatrix}
		a \\ b
	\end{bmatrix} +
	\begin{bmatrix}
		c \\ d
	\end{bmatrix} =
	\begin{bmatrix}
		a + c \\ b + d
	\end{bmatrix}
	\end{equation*}
	
	\subsection{Dot Product}
	
	\begin{defn}
		The dot product of $\vec{a}$ and $\vec{b}$ is the sum of their element-wise product: $$\vec{a} * \vec{b} = \sum_{i=0}^n \vec{a}_i \vec{b}_i$$
	\end{defn}
	
	Geometrically, the product of two vector is the product of their magnitude and the $\cos$ of their angle: $$\vec{a} * \vec{b} = \mo{\vec{a}} \mo{\vec{b}} \cos \theta$$
	
	Consider vector $\vec{a} * \vec{a}$, the resulting product should be   $\mov{a}^2$, as $\cos 0$ is $1$. Considering a triangle with angles $a$, $b$ and $c$ with angle $\theta$ opposing side $c$, the above can be deduced from the law of $\cos$: $$\mov{c}^2 = \mov{a}^2 + \mov{b}^2 - 2\mo{a}\mo{b}\cos \theta$$
	
	Since $\vec{c} = \vec{a} - \vec{b}$, it can be deduced that:
	\begin{align*}
		\mov{c}^2 = \vec{c} * \vec{c} &= (\vec{a} - \vec{b})*(\vec{a} - \vec{b}) \\
		&= \vec{a} * \vec{a} - \vec{a} * \vec{b} - \vec{b} * \vec{a} + \vec{b} * \vec{b}\\
		&= \mov{a}^2 + \mov{b}^2 - 2\vec{a}\vec{b}
	\end{align*}
	
	As shown in the last line of the transformation of $\mov{c}^2$, $-2\vec{a}\vec{b} = -2\mov{a}\mov{b} \cos \theta$.
	
	\subsection{Applications of Dot Product}
	
	Dot product can be used to compute lengths and angles. Consider a $\RR^3$ space with $P = (1, 0, 0)$, $Q = (0, 1, 0)$ and $R = (0, 0, 2)$, the angle $\angle RPQ$ can be found with dot product:
	\begin{align*}
		\overrightarrow{PR} * \overrightarrow{PQ} &= \mo{\overrightarrow{PR}} \mo{\overrightarrow{PQ}} \cos \theta \\
		\cos \theta &= \frac{\overrightarrow{PQ} * \overrightarrow{PR}}{\mo{\overrightarrow{PQ}}\mo{\overrightarrow{PR}}}
	\end{align*}
	
	By plugging in our example, we obtain:
	\begin{gather*}
		\cos \theta = \frac{1 + 0 + 0}{\sqrt{2} * \sqrt{5}} = \frac{1}{\sqrt{10}}
	\end{gather*}
	
	Another application of dot product is to determine the orthogonality of two vectors, i.e. when are two vectors $a$ and $b$ perpendicular. Note that the sign of a dot product denotes the directional relation of the two vectors $a$ and $b$ with angle $\theta$:
	\begin{itemize}
		\item $\vec{a} * \vec{b} > 0$ if $\theta < 90$
		\item $\vec{a} * \vec{b} = 0$ if $\theta = 90$
		\item $\vec{a} * \vec{b} > 0$ if $\theta > 90$
	\end{itemize}
	
	Consider the linear equation $x + 2y + 3z = 0$. The above equation can be written in the form of a dot product:
	\begin{gather*}
		\begin{bmatrix}
			1 \\ 2 \\ 3
		\end{bmatrix} *
		\begin{bmatrix}
			x \\ y \\ z
		\end{bmatrix} = 0
	\end{gather*}
	Therefore, the vector $\iv{x, y, z}$ is perpendicular to $\iv{1, 2, 3}$, making the former a plane, as $\cos \theta = 0$ where $\theta$ is the angle between the two vectors.
	
	\section{Cross Product}
	
	To get the component $a_u$ of $\vec{a}$ along unit vector $\vec{u}$:
	\begin{align*}
		a_u &= \mov{a}\cos{\theta} \\
		&= \mov{a}\mov{u}\cos{\theta} \\
		&= \vec{a} * \vec{u}
	\end{align*}
	
	\subsection{Geometric Interpretation of Determinants}
	
	To get the area of triangle $A$ with two edges denoted as vector $\vec{a}$ and $\vec{b}$ originating from point $O$, the area of $A$ is:
	\begin{gather*}
		\text{Area}(A) = \frac{1}{2} \mov{a}\mov{b} \sin \theta
	\end{gather*}
	
	The above could be obtained with dot product considering the complement angle and $\vec{a'}$ as $\vec{a}$ after $90^\circ$ rotation. $\angle a'Oa = \theta'$:
	\begin{align*}
		\text{Area}(A) &= \frac{1}{2} \mov{a}\mov{b} \sin \theta \\
		&= \vec{a'} * \vec{b} \\
		&= \iv{-a_2, a_1} * \iv{b_1, b_2} \\
		&= a_1 b_2 - a_2 b_2
	\end{align*}
	
	Note that the result of the above transformation is equivalent to the determinant of the matrix:
	\begin{equation*}
	\begin{bmatrix}
		a_1 & a_2 \\ b_1 & b_2
	\end{bmatrix}
	\end{equation*}
	
	\subsection{Determinant in $R^3$}
	
	The determinant of matrix $A = \iv{\vec{A}, \vec{B}, \vec{C}}^\top$ is:
	\begin{align*}
		\mo{A} = a_1
		\begin{vmatrix}
			b_2 & b_3 \\ c_2 & c_3
		\end{vmatrix} - a_2
		\begin{vmatrix}
			b_1 & b_3 \\ c_1 & c_3
		\end{vmatrix} + a_3
		\begin{vmatrix} 
			b_1 & b_2 \\ c_1 & c_2
		\end{vmatrix} 
	\end{align*}
	
	The determinant of matrix $\iv{\vec{a}, \vec{b}, \vec{c}}$ is the volume of the parallelepiped formed by edges $\vec{a}$, $\vec{b}$ and $\vec{c}$.
	
	\subsection{Rotation of Vectors}
	
	Vector $\vec{v} = \iv{a, b}$ after $90^\circ$ rotation gives $\iv{-b, a}$. Similarly, a clockwise rotation of $\vec{v}$ in the clockwise direction gives $\iv{b, -a}$.
	
	\subsection{Cross Product's Geometric Interpretation}
	
	The cross product of $\vec{a}$ and $\vec{b}$ is a vector $\vec{a} \times \vec{b}$. The above cross product can be represented via the "determinant" of the matrix:
	\begin{equation*}
	\begin{bmatrix}
		i & j & k \\ a_1 & a_2 & a_3 \\ b_1 & b_2 & b_3
	\end{bmatrix}
	\end{equation*}
	
	The coefficients of $i$, $j$ and $k$ are the $x$, $y$ and $z$ component of the resulting vector respectively.
	
	Consider the parallelepiped $A$ formed by edges $\vec{a}$, $\vec{b}$ and $\vec{c}$. Its volume is the area of the base scaled by the height $h_A$. We obtain:
	\begin{align*}
		\text{Area}(A) &= h_A * \mo{\vec{b} \times \vec{c}} \\
		&= \left(\vec{a} * \frac{\vec{b} \times \vec{c}}{\mo{\vec{b} \times \vec{c}}}\right) * \mo{\vec{b} \times \vec{c}} \\
		&= \vec{a} * (\vec{b} \times \vec{c})
	\end{align*}
	
	The above equation can be expanded via the definition of cross product to display its equality with the determinant of $\iv{\vec{a}, \vec{b}, \vec{c}}^\top$.
	
	\subsubsection{Right Hand Rule}
	
	The right hand rule is stupid. It doesn't work on people who can't tell left from right, like me. Soooooo basically given the cross product $\vec{e} = \vec{a} \times \vec{b}$ where $\vec{e}$ is in the position of poking your eyes out, $\vec{a}$ and $\vec{b}$ are positioned in a counter-clockwise manner.
	
	\section{Matrices}
	
	First, trivially: $$\vec{a} \times \vec{b} = -\vec{b} \times \vec{a}$$
	
	\subsection{Application of Cross Product}
	
	Give $p_1$, $p_2$ and $p_3$ in $\RR^3$ aligned along a hyperplane, the equation for the hyperplane $P(x, y, z)$ denotes the condition for $x$, $y$ and $z$ of a new given point $p$ for it to be contained in the hyperplane.
	
	One approach is to consider if the vectors $\overrightarrow{p_1p_3}$, $\overrightarrow{p_1p_2}$ and $\overrightarrow{p_1p}$ are in the same plane. In order words: $$\det(\iv{\overrightarrow{p_1p}, \overrightarrow{p_1p_2}, \overrightarrow{p_1p_3}}) = 0$$.
	
	Another approach is to obtain the cross product of $\overrightarrow{p_1p_2}$ and $\overrightarrow{p_1p_3}$, and see if the direction of the resultant vector (normal vector) is perpendicular to $\overrightarrow{p_1p}$. In other words: $$\overrightarrow{p_1p} * (\overrightarrow{p_1p_2} \times \overrightarrow{p_1p_3}) =  0$$
	
	\subsection{Linear Relations}
	
	Matrices encapsulate linear transformations. This is useful in scenarios such as change of basis, etc.
	
	Consider a change of basis from $\iv{x_1, x_2, x_3}$ to $\iv{u_1, u_2, u_3}$:
	\begin{align*}
		u_1 &= 2x_1 + 3x_2 + 3x_3 \\
		u_2 &= 2x_1 + 4x_2 + 5x_3 \\
		u_3 &= x_1 + x_2 + 2x_3
	\end{align*}
	
	The above transformation can be represented via a matrix:
	\begin{equation*}
		\begin{bmatrix}
			2 & 3 & 3 \\ 2 & 4 & 5 \\ 1 & 1 & 2
		\end{bmatrix} * \begin{bmatrix}
			x_1 \\ x_2 \\ x_3
		\end{bmatrix} = \begin{bmatrix}
			u_1 \\ u_2 \\ u_3
		\end{bmatrix}
	\end{equation*}
	
	The product of two matrices $A * X$ is the combinations of dot products between the rows of $A$ and the columns of $X$. Therefore, for dimension $A \in \RR^{m*n}$ and $X \in \RR^{n*o}$, $A*X \in \RR^{m*o}$.
	
\end{document}
