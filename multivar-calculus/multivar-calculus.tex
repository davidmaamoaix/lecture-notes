\documentclass[12pt]{article}

\usepackage{amsmath}
\usepackage{amssymb}
\usepackage{pgfplots}
\usepackage{mathtools}
\usepackage{booktabs}
\usepackage{indentfirst}

\usetikzlibrary{angles, quotes}

\pgfplotsset{compat=newest}

\title{Multivariable Calculus}
\author{Linxuan Ma}

\newcommand{\mo}[1]{\lvert #1 \rvert}
\newcommand{\mos}[1]{\lvert #1 \rvert^2}
\newcommand{\RR}{\mathbb{R}}
\newcommand{\p}{\partial}

\begin{document}
	\maketitle
	
	\abstract{Multivariable calculus is a useful concept in mathematical analysis. This document focuses on the differential and integral for multivariable calculus. The following notes are taken from MIT OCW $18.02$ Multivariable Calculus.}
	
	\section{Vectors}
	
	A vector $\vec{v}$ is a mathematical structure that has represents a tuple of direction and magnitude; its elements correspond to the cartesian coordinates of the represented point. A vector $\vec{v} \in \RR^n$ represents a point in n-dimension.
	
	The length $\mo{\vec{v}}$ of $\vec{v} \in \RR^n$ is a scalar that represents its magnitude: $$\mo{\vec{v}} = \sqrt{\sum_{i=0}^{n}(\vec{v}_i)^2}$$
	
	Note that a vector does not necessarily have a starting and ending point;  it merely represents an offset in a direction.
	
\end{document}
