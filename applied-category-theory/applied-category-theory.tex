\documentclass[12pt]{article}

\usepackage{amsmath}
\usepackage{amsthm}
\usepackage{amssymb}
\usepackage{indentfirst}
\usepackage{tikz-cd}
\usepackage{mathtools}

\theoremstyle{definition}
\newtheorem{defn}{Definition}[section]
\newtheorem{ex}{Exercise}[section]

\newcommand{\mo}[1]{\lvert #1 \rvert}
\newcommand{\mos}[1]{\lvert #1 \rvert^2}
\newcommand{\RR}{\mathbb{R}}
\newcommand{\T}{\text{true}}
\newcommand{\F}{\text{false}}

\title{Applied Category Theory}
\author{Linxuan Ma}

\begin{document}
	\maketitle
	
	\abstract{Category theory is the study of formalization of mathematical structures and concepts as an abstract directed graph. This note mainly consists of topics explored in \textit{An Invitation to Applied Category Theory}, with some bonus information from theoretical category theory and general abstract algebra to aid the understanding of certain concepts illustrated in the book.}
	
	\section{Generative Effects}
	
	A central focus of category theory is the abstraction of mathematical structures and their relationships. In particular, a mapping is a crucial structure in category theory that represents an abstracted relationship between two objects. A mapping $f: X \to Y$ denotes a relationship of $X$ with $Y$.
	
	Notice the notational resemblance of mappings to functions, which uses $f: X \to Y$ to denote a transformation from members of $X$ to members of $Y$ (i.e. a member of the dependent product formed over the constant dependent type $\prod_{(\_:A)}B$). However, in the scope of category theory, abstraction is imperative for the sake of generalizability, and therefore we do not limit mappings to represent concrete structures of transformation between objects. Such pattern of lacking concreteness is common among structures related to category theory and abstract algebra (oMg it literally has the word "abstract" in its name what were you expecting you dum dum).
	
	\begin{defn}
		A \emph{mapping} $f: X \to Y$ is an observation of $X$ via the relationship it has with $Y$, conforming to:
		\begin{itemize}
			\item \textbf{Composition}: given $f: A \to B$ and $g: B \to C$, there exists an associative composition $g \circ f: A \to C$
			\item \textbf{Identity}: for every mapping $f: A \to B$ there exists a unit mapping $\textbf{1}$ such that: $$f \circ \textbf{1}_A = f = \textbf{1}_B \circ f$$
		\end{itemize}
	\end{defn}
	
	In other words, mappings form a monoid over the composition operation.
	
	One noteworthy implication of mappings is that a mapping preserves some form of structure or information. To illustrate, a monotonous function $f$ preserves order: $x \leq y \iff f(x) \leq f(y)$, a metric-preserving function preserves the distance defined in the given system: $\mo{x - y} = \mo{f(x) - f(y)}$, etc. The specific information which the mapping preserve is determined by the category it belong to.
	
	\subsection{Introduction to Generative Effects}
	
	As mentioned, the notions of generative effect and structure preservation do not exist without a bounded context. We shall first construct a simple system to illustrate such ideas.
	
	Consider the set $U = \{A, B, C\}$ where a subset of $U$ denotes a transitive closure of elements in $U$. We now defined $\Phi(p)$ as the observation that if elements $A$ and $C$ are connected (potentially through a connection through other elements) in the closure described by subset $p$. Note that in this particular system, there is an isomorphism between two element's connectivity in the closure described by $p$ and their mutual existence in $p$; however, viewing the system in a way that is dependent on such behavior is discouraged, as the subset $p$ and the observation $\Phi$ are  meant to be methods of abstracting the idea of "connections".
	
	We then define the joining of two closures. Two points $x$ and $y$ are connected in the closure $\alpha \lor \beta$ is true if and only if there exists a sequence of point $\{z_1, z_2, \dots, z_n\} \in \alpha \cup \beta$ such that:
	\begin{itemize}
		\item $x$ is connected to $z_1$
		\item $z_i$ is connected to $z_{i + 1}$
		\item $z_n$ is connected to $y$
	\end{itemize}
	
	The above definition of join in our system may sound unnecessary for a system with only $3$ objects, yet it is useful if our system extends with more objects in the future.
	
	Now consider $p = \{A, B\}$ and $q = \{B, C\}$. Observe that the observation is false in both closure represented by $p$ and $q$:
	\begin{align*}
		\Phi(p) &= \text{false} \\
		\Phi(q) &= \text{false}
	\end{align*}
	
	However, $p \lor q$ connects $A$ and $C$ via the sequence of objects $\{B\} \in p \cup q$, resulting in: $$\phi(p \lor q) = \text{true}$$
	
	We observe that the observation $\Phi$ is not preserved over the join operation, i.e. $\Phi$ is inherently lossy with respect to join. This is a simple illustration of a generative effect: a property of a composite system that cannot be observed or determined given only its constituent subsystems.
	
	\subsection{Ordering}
	
	We now introduce more intricate structures based on the system of $U$ which we devised in the last section. An ordering system is a system whose objects can be compared with an object to yield a boolean result.
	
	To adapt to more sophisticated observations regarding $U$, we shall expand our representation of connected closures to a partition of $U$, where each element denotes a closure of elements in $U$. To perform $P \lor Q$, for each pair of connected elements $a$ and $b$ in $Q$, we add the union of the elements which $a$ and $b$ are located in $P$ to $P$, and remove the respective constituent subset.
	\begin{align*}
		P &= \{\{A, B\}, \{C\}\} \\
		Q &= \{\{A\}, \{B, C\}\} \\
		P \lor Q &= \{\{A, B, C\}\}
	\end{align*}
	
	Consider an order on $U$ where $P \preceq Q$ if for every connected pair $x$ and $y$ in $p$, $x$ and $y$ are connected in $Q$. For instance: $$\{\{A\}, \{B\}, \{C\}\} \preceq \{\{A, B\}, \{C\}\}$$
	
	The notion of $X \preceq Y$ can be viewed as a mapping: a relationship of $X$ with $Y$. Note that the relationship $\preceq$ does not occur on any two systems of $U$. This is equivalent to the fact that not all elements are comparable, as $\preceq$ is merely a mapping denoting a relationship between two elements; there is no guarantee of its existence (unless in the case of composition, as mentioned in its definition).
	
	By listing all possible systems of $U$ and representing $preceq$ with arrows, we obtain a \emph{Hasse diagram}:
	\begin{equation*}\begin{tikzcd}
		{} & \{\{A, B, C\}\} \\
		\{\{A, C\}, \{B\}\} \ar[ur] & \{\{A, B\}, \{C\}\} \ar[u] & \{\{A\}, \{B, C\}\} \ar[ul] \\
		{} & \{\{A\}, \{B\}, \{C\}\} \ar[ul] \ar[u] \ar[ur]
	\end{tikzcd}\end{equation*}
	
	Note that in $P \lor Q$, $P \preceq (P \lor Q)$ and $Q \preceq (P \lor Q)$. Interestingly, for any system $R$, $P \preceq R$ and $Q \preceq R$ implies that $(P \lor Q) \preceq R$. From the above diagram, we can conclude the formal definition of join:
	\begin{defn}
		A joined system $P \lor Q$ is the smallest system that is bigger than both $P$ and $Q$, and thereby satisfies:
		\begin{itemize}
			\item $P \preceq (P \lor Q)$
			\item $Q \preceq (P \lor Q)$
			\item $P \preceq R,\ Q \preceq R \iff (P \lor Q) \preceq R$
		\end{itemize}
	\end{defn}
	
	There is also an order on the boolean values $\mathbb{B} = \{\T, \F\}$: $\F \preceq \T$. In this context, the relationship $\preceq$ can be viewed as the term "implies". The joins of $\mathbb{B}$ are:
	\begin{align*}
		\T \lor \T &= \T \\
		\T \lor \F &= \T \\
		\F \lor \T &= \T \\
		\F \lor \F &= \F
	\end{align*}
	
	Note the similarity with the logical operator "or".
	The relationship $P \preceq Q$ is preserved over the aforementioned observation $\Phi$: $\Phi(P) \lor \Phi(Q) \preceq \Phi(P \lor Q)$.
	
	\subsubsection{Partitions and Equivalence}
	
	The partition of a set is useful in denoting a relationship between elements of the set, such as a closure under certain conditions. Formally, the definition of a partition is:
	\begin{defn}
		A \emph{partition} of $A$ is a set $P$, where each element $A_p$ is a nonempty subset of $A$, satisfying:
		\begin{itemize}
			\item $A = \bigcup_{p \in P} A_p$
			\item $p \neq q \Rightarrow A_p \cap A_q = \varnothing$
		\end{itemize}
	\end{defn}
	\begin{defn}
		An \emph{equivalence relation} on set $A$ is a binary relation $\sim$ that satisfies the following:
		\begin{itemize}
			\item \textbf{Reflexivity}: $\forall a \in A.\ a \sim a$
			\item \textbf{Symmetry}: $a \sim b \iff b \sim a$
			\item \textbf{Transitivity}: $a \sim b,\ b \sim c \Rightarrow a \sim c$
		\end{itemize}
	\end{defn}
	
	There exists a bijection between all partitions of a set $A$ and the equivalence relations on $A$.
	
	Observe that a equivalence relation $a \sim b$ can be defined as $a \in A_p$ and $b \in A_p$ given a partition $P$. In other words, $a$ is equivalent to $b$ if they are elements of the same part of a given partition. This satisfies the reflexivity ($a$ is always in the part as itself), symmetry and transitivity axioms.
	
	Dually, given an equivalence relation $\sim$ on set $A$, the set of transitive closures of $A$ in terms of $\sim$ forms a partition.
	
	Similarly, a partition of set $A$ can be expressed via a surjective function $f: A \to P$, where $P(a)$ denotes the part of the partition containing $a$ for $a \in A$. The pre-images $f^{-1}(p)$ for each $p \in P$ form a partition of $A$.
	
	\subsubsection{Preorders}
	
	As mentioned in the previous section, an ordering system is a system equipped with an order relation $\preceq$. The formal definition of an order relation is:
	\begin{defn}
		A \emph{preorder relation} on a set $A$ is a binary relation (denoted with $\preceq$) on the elements of $A$ that satisfies the following constraints:
		\begin{itemize}
			\item \textbf{Reflexivity}: $a \preceq a$ for all $a \in A$
			\item \textbf{Transitivity}: $a \preceq b,\ b \preceq c \Rightarrow a \preceq c$ for all $a, b, c \in A$
			\item \textbf{Equivalence}: $a \preceq b,\ b \preceq a \iff a \cong b$
		\end{itemize}
	\end{defn}
	
	The aforementioned set of booleans $\mathbb{B}$ forms a preorder with the relation $\F \preceq \T$:
	\begin{equation*}\begin{tikzcd}
		\T \ar[loop right] \\ \F \ar[u] \ar[loop left]
	\end{tikzcd}\end{equation*}
	
	Observe that a preorder $(A, =)$ can be defined on any set $A$. By defining the preorder relation as $=$, we enforce that a preorder relation $a \preceq b$ only occur when $a = b$. If $a \neq b$ then neither $a \preceq b$ nor $b \preceq a$ holds. A preorder satisfying the above is a \emph{discrete preorder}.
	
	The dual of discrete preorder is \emph{codiscrete preorder}, where for any $a, b \in A$ there exist relations $a \preceq b$ and $b \preceq a$. The formal definition of such preorder is a preorder with the total binary relation: $$A \times A \subseteq A \times A$$.
	
	\subsubsection{Partial Orders}
	
	\emph{Skeletality} refers to the requirement that $a \cong b$ implies $a = b$. A skeletal preorder is a \emph{partially ordered set}:
	\begin{defn}
		A \emph{partially ordered set} (poset) is a preorder $A$ with the additional constraint that $\forall a, b \in A.\ a \cong b \iff a = b$.
	\end{defn}
	
	Any discrete order already satisfies the constraint for a partially ordered set as there is no relation mapping between elements.
	
	Given the above definitions for orders and their representation in the form of Hasse diagrams, we can formalize the definition of a \emph{graph}:
	\begin{defn}
		A \emph{graph} $G = \{V, A, s, t\}$ is an abstract mathematical structure that encapsulates relations between elements. A graph consists of a set of vertices $V$ (often viewed as the elements of a set), a set of arrows $A$ (the relations/mappings between elements) and functions $s: A \to V$ (source function) and $t: A \to V$ (target function).
	\end{defn}
	
	A \emph{path} in a graph $G$ from $v_a$ to $v_b$ ($v_a, v_b \in V_G$) is any (potentially empty) sequence of composing arrows $\{a_1, a_2, \dots, a_n\}$ such that $s_G(a_1) = v_a$ and $t_G(a_n) = v_b$. Note the similarity in arrow composition and mapping composition.
	
	Any graph $G$ already establish a preorder relation between its vertices, as the relation can represent the existence of a path between two vertices in the graph, i.e. $a \to_G b \iff a \preceq b$ for $a, b \in V_G$. This satisfies the transitivity law due to the composition of arrows, and satisfies the reflexivity law due to the existence of a $0$-length path between a vertex and itself.
	
	A partial order does not enforce the existence of relation between any two objects; a partial order with a relation between any two elements is a \emph{total order}:
	\begin{defn}
		A \emph{total order} $A$ is a partial order that satisfies:
		\begin{itemize}
			\item For all $a, b \in A$, either $a \preceq b$ or $b \preceq a$. In order words, any two elements in $A$ is comparable.
		\end{itemize}
	\end{defn}
	
	\subsubsection{Monotone Maps}
	
	One significant relationship between preorders is \emph{monotone map}. A monotone map preserves ordering in a way.
	
	\begin{defn}
		A \emph{monotone map} $f: A \to B$ between preorders $(A, \preceq_A)$ and $(B, \preceq_B)$ satisfies:
		\begin{gather*}
			\forall x, y \in A.\ x \preceq_A y \Rightarrow f(x) \preceq_B f(y)
		\end{gather*}
	\end{defn}
	
	As an example, a map $f: \mathbb{N} \to \mathbb{B}$ that maps the upper set of $x \in \mathbb{N}$ to $\T$ and other elements to $\F$ is a monotone map.
	
	\begin{ex}
		The set $\uparrow p \coloneqq \{p'\in P \mid p \preceq p'\}$ is an upper set for any $p \in P$. This can be proven via considering
	\end{ex}

	
\end{document}
