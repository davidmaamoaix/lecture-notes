\documentclass[12pt]{article}

\usepackage{amsmath}
\usepackage{amssymb}
\usepackage{indentfirst}

\title{Complex Analysis}
\author{Linxuan Ma}

\begin{document}
	\maketitle
	
	\abstract{Convex analysis is the domain of mathematics that investigates functions of complex numbers. Despite its frequent occurrence in applied science, IB Mathematics seems to decide on not covering anything other than "haha $i^2$ is $-1$", so here's my attempt at creating notes for it after binging an entire semester of complex analysis in one night and getting sick immediately the next day due to sleep deprivation. Balanced IB life.}
	
	\section{Fundamentals}
	A complex number $x \in \mathbb{C}$ is a number of the form $$a + bi$$ where $a$ and $b$ are both real numbers.
	
	\subsection{Definitions}
	The imaginary number $i$ is defined as $\sqrt{-1}$. Due to its equivalence with $\sqrt{1}$, $i$ obeys all arithmetic laws that apply to root terms. Similarly, it coerces into $-1 \in \mathbb{R}$ in the case of $i^2$.
	
	Due to the encapsulation of $x \in \mathbb{C}$ over real numbers $a$ and $b$, a complex function $f: \mathbb{C} \to \mathbb{C}$ is isomorphic to $g: \mathbb{R}^2 \to \mathbb{R}^2$. Trivially, by decomposition we obtain
	\begin{gather*}
		\Re: \mathbb{C} \to \mathbb{R} \\
		\Im: \mathbb{C} \to \mathbb{R}
	\end{gather*}
	corresponding to retrieving real components $a$ and $b$.
	
	Deriving from $x \in \mathbb{C}$'s correspondence to a vector $\vec{x} \in \mathbb{R}$, the representation of a 2-dimensional coordinate can be concluded from $x$:
	
	\begin{equation*}
		\begin{bmatrix}
			\Re(x) \\ \Im(x)
		\end{bmatrix}
	\end{equation*}
	
	From such geometric representation, further vector-like properties can be defined (conjugate, modulus and argument) for $c \in \mathbb{C}$ with real part $a$ and imaginary part $b$:
	
	\begin{gather*}
		\overline{c} = a - bi \\
		\lvert c \rvert = \sqrt{a^2 + b^2} \\
		\arg c = \arctan \frac{b}{a}
	\end{gather*}
	
	Note the isomorphism of the modulus-argument pair with the standard $a + bi$ form under the domain $-\pi < \theta \leq \pi$ for argument. In the remaining portion of this note, a complex number may take any of the following form:
	
	\begin{enumerate}
		\item Regular form: $a + bi$ where $a, b \in \mathbb{R}$
		\item Tuple form (equivalent to $a + bi$): $(a, b)$
		\item Polar form (modulus $r$ and argument $\theta$): $r(cos \theta + i \sin \theta)$
	\end{enumerate}
	
	\subsection{Complex Arithmetic}
	
	Trivially, due to the nature of square roots:
	
	\begin{gather*}
		(a, b) + (c, d) = (a + c, b + d) \\
		(a, b) - (c, d) = (a - c, b - d) \\
		(a, b) * (c, d) = (ac - bd, ad + bc)
	\end{gather*}
	
	Conjugation is distributive over addition, subtraction, multiplication and division:
	
	\begin{gather*}
		\overline{z + w} = \overline{z} * \overline{w} \\
		\overline{z - w} = \overline{z} * \overline{w} \\
		\overline{z * w} = \overline{z} * \overline{w} \\
		\overline{\left(\frac{z}{w}\right)} = \frac{\overline{z}}{\overline{w}}
	\end{gather*}
	
	Rules regarding the polar representation can be trivially obtained via substitution of definitions:
	
	\begin{gather*}
		\lvert z * w \rvert = \lvert z \rvert * \lvert w \rvert \\
		\arg(z * w) = \arg z + \arg w
	\end{gather*}
	
	
	
\end{document}