\documentclass[12pt]{article}

\usepackage{amsmath}
\usepackage{amssymb}
\usepackage{indentfirst}

\title{Complex Analysis}
\author{Linxuan Ma}

\begin{document}
	\maketitle
	
	\abstract{Convex analysis is the domain of mathematics that investigates functions of, or consisting of, complex numbers. Despite its frequent occurrence in applied science, IB Mathematics seems to decide on not covering anything other than "haha $i^2$ is $-1$", so here's my attempt at creating notes for it after binging an entire semester of complex analysis in one night and getting sick immediately the next day due to sleep deprivation. Balanced IB life.}
	
	\section{Fundamentals}
	A complex number $x \in \mathbb{C}$ is a number of the form $$a + bi$$ where $a$ and $b$ are both real numbers.
	
	
	The imaginary number $i$ is defined as $\sqrt{-1}$. Due to its equivalence with $\sqrt{1}$, $i$ obeys all arithmetic laws that apply to root terms. Similarly, it coerces into $-1 \in \mathbb{R}$ in the case of $i^2$.
	
	Due to the encapsulation of $x \in \mathbb{C}$ over real numbers $a$ and $b$, a complex function $f: \mathbb{C} \to \mathbb{C}$ is isomorphic to $g: \mathbb{R}^2 \to \mathbb{R}^2$. Trivially, by decomposition we obtain
	\begin{gather*}
		\Re: \mathbb{C} \to \mathbb{R} \\
		\Im: \mathbb{C} \to \mathbb{R}
	\end{gather*}
	corresponding to retrieving real components $a$ and $b$.
	
	\subsection{Axioms}
	
\end{document}