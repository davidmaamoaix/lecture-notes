\documentclass[12pt]{article}

\usepackage{amsmath}
\usepackage{amssymb}
\usepackage{indentfirst}

\title{Complex Analysis}
\author{Linxuan Ma}

\begin{document}
	\maketitle
	
	\newcommand{\mo}[1]{\lvert #1 \rvert}
	\newcommand{\mos}[1]{\lvert #1 \rvert^2}
	
	\abstract{Convex analysis is the domain of mathematics that investigates functions of complex numbers. Despite its frequent occurrence in applied science, IB Mathematics seems to decide on not covering anything other than "haha $i^2$ is $-1$", so here's my attempt at creating notes for it after binging an entire semester of complex analysis in one night and getting sick immediately the next day due to sleep deprivation. Balanced IB life.}
	
	\section{Fundamentals}
	A complex number $x \in \mathbb{C}$ is a number of the form $$a + bi$$ where $a$ and $b$ are both real numbers.
	
	\subsection{Definitions}
	The imaginary number $i$ is defined as $\sqrt{-1}$. Due to its equivalence with $\sqrt{1}$, $i$ obeys all arithmetic laws that apply to root terms. Similarly, it coerces into $-1 \in \mathbb{R}$ in the case of $i^2$.
	
	Due to the encapsulation of $x \in \mathbb{C}$ over real numbers $a$ and $b$, a complex function $f: \mathbb{C} \to \mathbb{C}$ is isomorphic to $g: \mathbb{R}^2 \to \mathbb{R}^2$. Trivially, by decomposition we obtain
	\begin{gather*}
		\Re: \mathbb{C} \to \mathbb{R} \\
		\Im: \mathbb{C} \to \mathbb{R}
	\end{gather*}
	corresponding to retrieving real components $a$ and $b$.
	
	Deriving from $x \in \mathbb{C}$'s correspondence to a vector $\vec{x} \in \mathbb{R}$, the representation of a 2-dimensional coordinate can be concluded from $x$:
	
	\begin{equation*}
		\begin{bmatrix}
			\Re(x) \\ \Im(x)
		\end{bmatrix}
	\end{equation*}
	
	From such geometric representation, further vector-like properties can be defined (conjugate, modulus and argument) for $c \in \mathbb{C}$ with real part $a$ and imaginary part $b$:
	
	\begin{gather*}
		\overline{c} = a - bi \\
		\mo{c} = \sqrt{a^2 + b^2} \\
		\arg c = \arctan \frac{b}{a}
	\end{gather*}
	
	Note the isomorphism of the modulus-argument pair with the standard $a + bi$ form under the domain $-\pi < \theta \leq \pi$ for argument. In the remaining portion of this note, a complex number may take any of the following form:
	
	\begin{enumerate}
		\item Regular form: $a + bi$ where $a, b \in \mathbb{R}$
		\item Tuple form (equivalent to $a + bi$): $(a, b)$
		\item Polar form (modulus $r$ and argument $\theta$): $r(cos \theta + i \sin \theta)$
	\end{enumerate}
	
	\subsection{Complex Arithmetic}
	
	Trivially, due to the nature of square roots:
	
	\begin{gather*}
		(a, b) + (c, d) = (a + c, b + d) \\
		(a, b) - (c, d) = (a - c, b - d) \\
		(a, b) * (c, d) = (ac - bd, ad + bc) \\
		\frac{(a, b)}{(c, d)} = \frac{(a, b)(c, -d)}{(c, d)(c, -d)}
	\end{gather*}
	
	Conjugation is distributive over addition, subtraction, multiplication and division:
	
	\begin{gather*}
		\overline{z + w} = \overline{z} * \overline{w} \\
		\overline{z - w} = \overline{z} * \overline{w} \\
		\overline{z * w} = \overline{z} * \overline{w} \\
		\overline{\left(\frac{z}{w}\right)} = \frac{\overline{z}}{\overline{w}}
	\end{gather*}
	
	It is clear that the complex conjugation $z \to \overline z$ is an automorphism of $\mathbb{C}$ (an isomorphic endofunctor).
	
	Trivially, $z * \overline z \in \mathbb{R}$.
	
	Rules regarding the polar representation can be trivially obtained via substitution of definitions:
	
	\begin{gather*}
		\mo{z * w} = \mo{z} * \mo{w} \\
		\arg(z * w) = \arg z + \arg w
	\end{gather*}
	
	Modulus can be viewed as the distance of the represented point to the origin, and thereby the rule:
	
	\begin{gather*}
		\mo{z * w} \leq \mo{z} * \mo{w}
	\end{gather*}
	
	\subsection{Simple Applications}
	
	The following section explores rudimentary applications we can derive from the previously covered rules of complex numbers.
	
	\subsubsection{Magma of Sums of Squares Under Multiplication}
	
	Consider the number theory problem:
	\begin{center}
		Which integers are sums of two squares?
	\end{center}
	
	With complex numbers, we realize that the integers that are the sum of two squares form a magma under multiplication.
	\\\\
	\textbf{Proof.} There exists an rearrangement of terms in a product of sums of 2 squares such that: $$(a^2 + b^2)(c^2 + d^2) \to (ac - bd)^2 (ad + bc)^2$$
	The above transformation is derived from the complex number rules related to complex modulus:
	
	\begin{gather*}
		a^2 + b^2 = \mos{a + ib} \\
		c^2 + d^2 = \mos{c + id}
	\end{gather*}
	Therefore $(a^2 + b^2)(c^2 + d^2)$ can be written as:
	
	\begin{align*}
		  & (a^2 + b^2)(c^2 + d^2) \\
		= & \mos{(a + ib)(c + id)} \\
		= & \mos{ac - bd + i(ad + bc)} \\
		= & (ac - bd)^2 + (ad + bc)^2
	\end{align*}
	 
	Extending from the proof above, consider two given sum of squares:
	
	\begin{gather*}
		5 = 1^2 + 2^2 \\
		13 = 2^2 + 3^2
	\end{gather*}
	
	By multiplication we obtain: $$5 * 13 = 65$$
	
	Note that $65$ can also be represented as different sums of two squares: $1^2 + 8^2 = 4^2 + 7^2 = 65$. By writing the constituent of $5$ and $13$ as complex modulus, we obtain:
	
	\begin{gather*}
		\mos{1 + 2i} = 5 \\
		\mos{2 + 3i} = 13
	\end{gather*}
	
	By multiplying $1 + 2i$ and $2 + 3i$ and conjugating one of them, we obtain:
	
	\begin{gather*}
		(1 + 2i)(2 + 3i) = -4 + 7i \\
		(1 + 2i)(2 - 3i) = 8 + i
	\end{gather*}
	corresponding to the other two solution $1^2 + 8^2 = 65$ and $4^2 + 7^2 = 65$.
	
	\subsubsection{Pythagorean Triples}
	
	Another application of complex arithmetic relates to the Pythagorean theorem, easily generating a set of Pythagorean triples from complex numbers. Consider the Pythagorean theorem of edge $a, b, c$ where $c \leq a + b$ in a right triangle: $$a^2 + b^2 = c^2$$
	
	Suppose $a + bi = (x + yi) ^2$ where $x, y \in \mathbb{N}$, then:
	\begin{gather*}
		\mo{a + bi} = \mo{(x + yi)^2} \\
		a^2 + b^2 = \mos{(x + yi)^2} = (x^2 + y^2)^2
	\end{gather*}
	
	We can then substitute $x$ and $y$ with any natural number and apply complex modulus properties in:
	\begin{gather*}
		\alpha = (x + yi)^2 \\
		\beta = \mos{x + yi}
	\end{gather*}
	
	$a$, $b$ and $c$ can then be obtained by:
	\begin{gather*}
		a = \Re(\alpha) \\
		b = \Im(\alpha) \\
		c = \beta
	\end{gather*}
	
	\subsubsection{Quaternions}
	
	A feasible expansion of the complex number system into $\mathbb{R}^3$ has yet to be proposed. However, there exists a $\mathbb{R}^4$ expansion, the quaternions, of the complex plane.
	
	In the quaternion system, a complex number $a + bi$ is expanded into $a + bi + cj + dk$, where the multiplication of $i$, $j$ and $k$ is not communicative:
	\begin{gather*}
		i^2 = j^2 = k^2 = ijk = -1 \\
		ij = k = -ji \\
		jk = i = -kj \\
		ki = j = -ik
	\end{gather*}
	
	Similar to finding the inverse of a complex number $a + bi$ (multiplying nominator and denominator by its conjugate):
	$$\frac{1}{a + bi} = \frac{a - bi}{a^2 + b^2}$$
	, there also exists an inverse for a quaternion:
	$$\overline{a + bi + cj + dk} = a - bi - cj - dk$$
	
	The product of a quaternion with its conjugate is:
	\begin{align*}
		z\overline{z} = a^2 + b^2 + c^2 + d^2
	\end{align*}
	
	Note that the cross terms (such as $bcij$ and $bcji$) in the above equation vanish due to the non-communitivity ($ij$ = $-ji$).
	
	The inverse of a quaternion can thus be found:
	$$\frac{1}{a + bi + cj + dk} = \frac{a - bi - cj - dk}{a^2 + b^2 + c^2 + d^2}$$
	
	All none-zero quaternions have inverses, similar to complex numbers.
	
\end{document}