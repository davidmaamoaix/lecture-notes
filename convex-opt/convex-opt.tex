\documentclass[12pt]{article}

\usepackage{amsmath}
\usepackage{amssymb}
\usepackage{amsthm}
\usepackage{pgfplots}
\usepackage{mathtools}
\usepackage{booktabs}
\usepackage{indentfirst}
\usepackage{hyperref}

\usetikzlibrary{angles, quotes}

\pgfplotsset{compat=newest}

\title{Convex Optimization}
\author{Linxuan Ma}

\newcommand{\mo}[1]{\lvert #1 \rvert}
\newcommand{\mos}[1]{\lvert #1 \rvert^2}
\newcommand{\mov}[1]{\lvert \vec{#1} \rvert}
\newcommand{\RR}{\mathbb{R}}
\newcommand{\CC}{\mathbb{C}}
\newcommand{\p}{\partial}
\newcommand{\iv}[1]{\langle #1 \rangle}
\newcommand{\adj}{\text{adj}}
\newcommand{\dom}{\text{dom}}
\newcommand{\st}{\text{s.t. }}

\theoremstyle{definition}
\newtheorem{defn}{Definition}[section]
\newtheorem{ex}{Exercise}

\begin{document}
	\maketitle
	
	\abstract{Convex optimization is a subset of mathematical optimizations that centers around optimization of convex functions over convex sets. The following note is taken from CMU 10-725 (watched from YouTube) as well as the book \emph{Convex Optimization} (Boyd. et al)}. My attempt at relevant programs and exercises can be found at \url{https://github.com/davidmaamoaix/convex-opt}.
	
	\section{Introduction}
	
	
\end{document}