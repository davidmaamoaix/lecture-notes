\documentclass[12pt]{article}

\usepackage{amsmath}
\usepackage{amsthm}
\usepackage{amssymb}
\usepackage{indentfirst}
\usepackage{tikz-cd}
\usepackage{mathtools}
\usepackage[shortlabels]{enumitem}
\usepackage{xcolor}
\usepackage[margin=0.75in, top=1in, a4paper]{geometry}

\theoremstyle{definition}
\newtheorem{defn}{Definition}[section]
\newtheorem{ex}{Exercise}[section]
\newtheorem{oex}[ex]{*Exercise}

\newcommand{\mo}[1]{\lvert #1 \rvert}
\newcommand{\mos}[1]{\lvert #1 \rvert^2}
\newcommand{\RR}{\mathbb{R}}
\newcommand{\QQ}{\mathbb{Q}}
\newcommand{\NN}{\mathbb{N}}
\newcommand{\T}{\text{true}}
\newcommand{\F}{\text{false}}
\newcommand{\mov}[1]{\lvert \vec{#1} \rvert}
\newcommand{\CC}{\mathbb{C}}
\newcommand{\p}{\partial}
\newcommand{\iv}[1]{\langle #1 \rangle}
\newcommand{\adj}{\text{adj}}
\newcommand{\dom}{\text{dom}}
\newcommand{\st}{\text{s.t. }}
\newcommand{\ltc}[1]{(\colorbox{lightgray}{\textbackslash #1})}
\newcommand{\bras}[1]{\lbrace #1 \rbrace}

\definecolorset{gray/rgb/hsb/cmyk}{}{}%
 {black,0/0,0,0/0,0,0/0,0,0,1;%
  darkgray,.25/.25,.25,.25/0,0,.25/0,0,0,.75;%
  gray,.5/.5,.5,.5/0,0,.5/0,0,0,.5;%
  lightgray,.85/.85,.85,.85/0,0,.85/0,0,0,.15;%
  white,1/1,1,1/0,0,1/0,0,0,0}

\title{\vspace{-2.0cm}Math Symbols Cheatsheet}
\author{Linxuan Ma}

\begin{document}
	\maketitle
	
	\section{Sets}
	
	A set is a collection of objects, which are referred to as the $members$ or $elements$ of the set.
	
	\begin{itemize}
		\item $a \in A$: $a$ is an element of the set $A$.
		\item $|A|$: The cardinality (i.e. size) of the set $A$, e.g. the set $\bras{1, 3, 5}$ has cardinality $3$.
	\end{itemize}
	
	\subsection{Common Sets}
	\begin{itemize}
		\item $\varnothing$: The empty set with no elements.
		\item $\NN$: The set of natural numbers, e.g. $2$, $6969$, $42$.
		\item $\mathbb{Z}$: The set of integers, e.g. $-2, 0, 9$.
		\item $\mathbb{Q}$: The set of rational numbers, e.g. $2.34$, $\frac{6}{7}$.
		\item $\RR$: The set of real numbers, e.g. $12$, $\frac{3}{8}$, $-\sqrt{2}$.
		\item $\RR_+$: The set of non-negative real numbers, e.g. $0$, $\pi$, $1$. This notation works for other number sets (e.g. $Z_+$ denotes non-negative integers).
		\item $\RR_{++}$: The set of strictly positive real numbers (not including $0$).
		\item $\CC$: The set of complex numbers, e.g. $2 + 3i$, $5i$, $12$.
		\item $\RR^n$: The set of vectors of length (dimension) $n$.
		\item $\RR^{m \times n}$: The set of matrices of size $(m, n)$.
	\end{itemize}
	
	\subsection{Construction of Sets}
	
	An arbitrary set $S$ can be constructed in numerous ways:
	\begin{itemize}
		\item $S = \bras{a, b, c}$: Set $S$ has elements $a$, $b$ and $c$.
		\item $S = \bras{x \mid x \in A}$: Set $S$ comprise of elements from set $A$.
		\item $S = \bras{x \mid x \in A,\ C_1(x), \dots, C_n(x)}$: Set $S$ comprise of all elements from set $A$ that satisfies all assertion $C_1, \dots, C_n$ of the element, e.g. $\bras{x \mid x \in \NN,\ x > 5}$ is the set of all natural numbers that are larger than $5$.
		\item $S = \bras{f(a, b) \mid a \in A,\ b \in B}$: Set $S$ contains the results of applying function $f$ to all combinations of elements from set $A$ and $B$.
	\end{itemize}
	
	Apart from the first notation, the rest are referred to as \emph{set comprehensions}. They can be used in conjunction, i.e. a set comprehension can contain multiple assertion statements and multiple element definition statements.
	
	\subsection{Set Operations}
	
	By operating on sets, new sets can be formed from existing sets.
	
	\begin{itemize}
		\item $\overline{A}$: The complementary set of $A$. The definition of this is sensitive to the definition of the universal set in the current context.
		\item $A \setminus B$: The difference of $A$ and $B$, i.e. all members of $A$ that are not contained in $B$.
		\item $A \cap B$: The intersection of $A$ and $B$, i.e. the set whose elements are all elements contained in \emph{both} $A$ and $B$.
		\item $A \cup B$: The union of $A$ and $B$, i.e. the set whose elements are all elements contained in \emph{either} $A$ or $B$.
		\item $A \sqcup B$: The disjoint union of $A$ and $B$. This is similar to a union, except that each element in the resulting union has semantics indicating which source set (of $A$ and $B$) this element came from. One representation of this idea is treating $\bras{(a, A) \mid a \in A} \cup \bras{(b, B) \mid b \in B}$ as the disjoint union of $A$ and $B$.
		\item $A \times B$: The cartesian product of $A$ and $B$. This is the set of tuples/pairs defined as $\bras{(a, b) \mid a \in A,\ b \in B}$.
	\end{itemize}
	
	It is clear that the cardinality of some composed set can be inferred from its constituent sets. This is useful when proving theorems via the Howard-Curry isomorphism.
	\begin{itemize}
		\item $A \cap B = \varnothing \text{ ($A$ and $B$ are disjoint)} \iff |A \cap B| = |A| + |B|$
		\item $|A \sqcup B| = |A| + |B|$
		\item $|A \times B| = |A| * |B|$
	\end{itemize}
	
\end{document}
