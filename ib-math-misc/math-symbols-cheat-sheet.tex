\documentclass[12pt]{article}

\usepackage{amsmath}
\usepackage{amsthm}
\usepackage{amssymb}
\usepackage{indentfirst}
\usepackage{tikz-cd}
\usepackage{mathtools}
\usepackage[shortlabels]{enumitem}
\usepackage{xcolor}
\usepackage{hyperref}
\usepackage{multicol}
\usepackage[margin=0.75in, top=1in, a4paper]{geometry}

\theoremstyle{definition}
\newtheorem{defn}{Definition}[section]
\newtheorem{ex}{Exercise}[section]
\newtheorem{oex}[ex]{*Exercise}

\newcommand{\mo}[1]{\lvert #1 \rvert}
\newcommand{\mos}[1]{\lvert #1 \rvert^2}
\newcommand{\RR}{\mathbb{R}}
\newcommand{\QQ}{\mathbb{Q}}
\newcommand{\NN}{\mathbb{N}}
\newcommand{\T}{\text{true}}
\newcommand{\F}{\text{false}}
\newcommand{\mov}[1]{\lvert \vec{#1} \rvert}
\newcommand{\CC}{\mathbb{C}}
\newcommand{\p}{\partial}
\newcommand{\iv}[1]{\langle #1 \rangle}
\newcommand{\adj}{\text{adj}}
\newcommand{\dom}{\text{dom}}
\newcommand{\st}{\text{s.t. }}
\newcommand{\ltc}[1]{\colorbox{lightgray}{\textbackslash #1}}
\newcommand{\ltcc}[1]{\colorbox{lightgray}{#1}}
\newcommand{\bras}[1]{\lbrace #1 \rbrace}
\newcommand{\braks}[1]{\lbrack #1 \rbrack}
\newcommand{\id}[1]{\text{id}_{#1}}

\definecolorset{gray/rgb/hsb/cmyk}{}{}%
 {black,0/0,0,0/0,0,0/0,0,0,1;%
  darkgray,.25/.25,.25,.25/0,0,.25/0,0,0,.75;%
  gray,.5/.5,.5,.5/0,0,.5/0,0,0,.5;%
  lightgray,.85/.85,.85,.85/0,0,.85/0,0,0,.15;%
  white,1/1,1,1/0,0,1/0,0,0,0}

\title{\vspace{-2.0cm}Mathematics Notation Cheatsheet}
\author{David Ma}

\begin{document}
	\maketitle
	
	\section{Sets}
	
	A set is a collection of objects, which are referred to as the $members$ or $elements$ of the set.
	
	\begin{itemize}
		\item $a \in A$: $a$ is an element of the set $A$.
		\item $|A|$: The cardinality (i.e. size) of the set $A$, e.g. the set $\bras{1, 3, 5}$ has cardinality $3$.
	\end{itemize}
	
	\subsection{Common Sets}
	\begin{itemize}
		\item $\varnothing$: The empty set with no elements.
		\item $\NN$: The set of natural numbers, e.g. $2$, $6969$, $42$.
		\item $\mathbb{Z}$: The set of integers, e.g. $-2, 0, 9$.
		\item $\mathbb{Q}$: The set of rational numbers, e.g. $2.34$, $\frac{6}{7}$.
		\item $\RR$: The set of real numbers, e.g. $12$, $\frac{3}{8}$, $-\sqrt{2}$.
		\item $\RR_+$: The set of non-negative real numbers, e.g. $0$, $\pi$, $1$. This notation works for other number sets (e.g. $Z_+$ denotes non-negative integers).
		\item $\RR_{++}$: The set of strictly positive real numbers (not including $0$).
		\item $\CC$: The set of complex numbers, e.g. $2 + 3i$, $5i$, $12$.
		\item $\RR^n$: The set of vectors of length (dimension) $n$.
		\item $\RR^{m \times n}$: The set of matrices of size $(m, n)$.
	\end{itemize}
	
	\subsection{Construction of Sets}
	
	An arbitrary set $S$ can be constructed in numerous ways:
	\begin{itemize}
		\item $S = \bras{a, b, c}$: Set $S$ has elements $a$, $b$ and $c$.
		\item $S = \bras{x \mid x \in A}$: Set $S$ comprise of elements from set $A$.
		\item $S = \bras{x \mid x \in A,\ C_1(x), \dots, C_n(x)}$: Set $S$ comprise of all elements from set $A$ that satisfies all assertion $C_1, \dots, C_n$ of the element, e.g. $\bras{x \mid x \in \NN,\ x > 5}$ is the set of all natural numbers that are larger than $5$.
		\item $S = \bras{f(a, b) \mid a \in A,\ b \in B}$: Set $S$ contains the results of applying function $f$ to all combinations of elements from set $A$ and $B$.
	\end{itemize}
	
	Apart from the first notation, the rest are referred to as \emph{set comprehensions}. They can be used in conjunction, i.e. a set comprehension can contain multiple assertion statements and multiple element definition statements.
	
	\subsection{Set Operations}
	
	By operating on sets, new sets can be formed from existing sets.
	
	\begin{itemize}
		\item $\overline{A}$: The complementary set of $A$. The definition of this is sensitive to the definition of the universal set in the current context.
		\item $A \setminus B$: The difference of $A$ and $B$, i.e. all members of $A$ that are not contained in $B$.
		\item $A \cap B$: The intersection of $A$ and $B$, i.e. the set whose elements are all elements contained in \emph{both} $A$ and $B$.
		\item $A \cup B$: The union of $A$ and $B$, i.e. the set whose elements are all elements contained in \emph{either} $A$ or $B$.
		\item $A \sqcup B$: The disjoint union of $A$ and $B$. This is similar to a union, except that each element in the resulting union has semantics indicating which source set (of $A$ and $B$) this element came from. One representation of this idea is treating $\bras{(a, A) \mid a \in A} \cup \bras{(b, B) \mid b \in B}$ as the disjoint union of $A$ and $B$.
		\item $A \times B$: The cartesian product of $A$ and $B$. This is the set of tuples/pairs defined as $\bras{(a, b) \mid a \in A,\ b \in B}$.
	\end{itemize}
	
	It is clear that the cardinality of some composed set can be inferred from its constituent sets. This is useful when proving theorems via the Howard-Curry isomorphism.
	\begin{itemize}
		\item $A \cap B = \varnothing \text{ ($A$ and $B$ are disjoint)} \iff |A \cap B| = |A| + |B|$
		\item $|A \sqcup B| = |A| + |B|$
		\item $|A \times B| = |A| * |B|$
	\end{itemize}
	
	In addition, note that there are also compact forms for some of the above operations:
	\begin{align*}
		\bigcap^n_{i=1} a_i &= a_1 \cap a_2 \cap \dots \cap a_n\\
		\bigcup^n_{i=1} a_i &= a_1 \cup a_2 \cup \dots \cup a_n
	\end{align*}
	
	\subsection{Set Relations}
	
	The relationship between sets can also be described with symbols:
	\begin{itemize}
		\item $A \subseteq B$: $A$ is a subset of $B$, i.e. all members of $A$ are also members of $B$.
		\item $A \not\subseteq B$: $A$ is not a subset of $B$, i.e. there exists at least one member of $A$ that is not a member of $B$.
		\item $A \subset B$: $A$ is a strict ($A \neq B$) subset of $B$.
		\item $A \not\subset B$: $A$ is not a strict subset of $B$.
 	\end{itemize}
 	
 	\section{Functions}
 	
 	The \emph{signature} of a function denotes what type of function it is, i.e. what are the parameters and return types of the function.
 	
 	To illustrate, the function signature of the addition function $+$ over the set of natural numbers can be written in the forms of either:
 	\begin{itemize}
 		\item $+ : \NN \to \NN \to \NN$
 		\item $+ : \NN \times \NN \to \NN$
 	\end{itemize}
 	
 	The above signatures state that "$+$ is a function that accepts two parameters from $\NN$ and returns a value of type $\NN$".
 	
 	Note that the two signatures listed above are equivalent up to isomorphism (over the currying operation). From this observation, it is evident that the arrow $\to$ in function signatures is right-associative.
 	
 	Some additional notations of functions:
 	\begin{itemize}
 		\item $\id{A}: A \to A$ is the identity function that simply returns the given parameter.
 		\item In addition to the common function definition, e.g. $f(x, y) = x^2 + y^2$, a function can also be defined in-line with the mapping arrow $\mapsto$, e.g. $(x, y) \mapsto x^2 + y^2$.
 		\item Given $f : A \to B$ and $g : B \to C$, the function composition $g \circ f : A \to C$ denotes the chaining of $g$ after $f$, i.e. $x \mapsto g(f(x))$.
 	\end{itemize}
 	
 	\subsection{Function Descriptions}
 	
 	A function $f : A \to B$ can be described according to its behavior on how it maps elements from $A$ to $B$. Specifically, $f$ can be:
 	\begin{enumerate}
 		\item \textbf{Injective}: $f(x) \neq f(y)$ for all $x, y \in A$, $x \neq y$.
 		\item \textbf{Surjective}: For all $b \in B$, there exists an $a \in A$ such that $f(a) = b$.
 		\item \textbf{Bijective}: A bijective function is one that is both injective and surjective.
 	\end{enumerate}
 	
 	Sets can also be described in relation to a function. Consider the function $f : X \to Y$, then a set in relation to $f$ can be described as:
 	\begin{itemize}
 		\item \textbf{Domain}: The domain of $f$ is $X$.
 		\item \textbf{Codomain}: The codomain of $f$ is $Y$.
 		\item \textbf{Image} (range): The image $M$ of $f$ is the set of actual values $f(x)$ for all $x \in X$. If $f$ is surjective, then $M = Y$.
 		\item \textbf{Preimage}: The preimage of $P \subseteq Y$ over $f$ is the set of $x \in X$ such that $f(x) \in P$.
 	\end{itemize}
 	
 	For example, the notions of image and preimage are commonly used when describing linear mapping of vector spaces, as singular matrices (i.e. whose determinant is $0$) maps a vector space to a strict subset of itself, thereby distinguishing its image from its codomain.
	
	\section{Relations}
	
	In academic writing, custom relationships between arbitrary objects are often defined for better elaboration. Such an abstraction is especially prevalent in proofs. This section aims to clarify some common relation definitions and terminologies.
	
	\begin{defn}
		A \emph{preorder} relation on set $A$ is a binary relation that denotes an order over the elements of $A$. A preorder relationship between two objects is denoted as $a \preceq b$ or with a similar operator.
		
		A preorder relation $\preceq$ on set $A$ must satisfy the following axioms:
		\begin{enumerate}
			\item \textbf{Identity}: $a \preceq a$ for all $a \in A$
			\item \textbf{Transitivity}: $a \preceq b \land b \preceq c \implies a \preceq c$
			\item \textbf{Anti-symmetry}: $a \preceq b \land b \preceq a \implies a \cong b$
		\end{enumerate}
		where $\cong$ is some notion of equivalence (see below).
 	\end{defn}
 	
 	For example, a preorder can be formed on the set of restaurants near YKPS based on their distance to the school: $a \preceq b$ if the distance to $a$ is less than or equal to that of $b$.
 	
 	\begin{defn}
 		An \emph{equivalence} relation $a \cong b$ states that $a$ and $b$ are equal in some sense. A definition for equivalence on set $A$ must satisfy the following constraints:
 		\begin{enumerate}
 			\item \textbf{Reflexivity}: $a \cong a$ for all $a \in A$
 			\item \textbf{Symmetry}: $a \cong b \iff b \cong a$
 			\item \textbf{Transitivity}: $a \cong b \land b \cong c \implies a \cong c$
 		\end{enumerate}
 	\end{defn}
 	
 	The symbol for equivalence can also be a tilde $\sim$.
 	
 	For example, for a purchasing problem, the equivalence of purchasing strategies can be defined as having the same cost/price.
 	
 	\begin{defn}
 		An \emph{isomorphism} between structures (structures are more generalized than sets) $A$ and $B$ states that "having either $A$ or $B$ is as good as having the other". Formally, structures $A$ and $B$ are isomorphic if there exists a pair of functions $f : A \to B$ and $g : B \to A$ such that:
 		\begin{enumerate}
 			\item $g \circ f = \id{A}$
 			\item $f \circ g = \id{B}$
 		\end{enumerate}
 	\end{defn}
 	
 	For example, a bijection is an isomorphism in the category of sets.
 	
 	\section{Formal Logic}
 	
 	Logic statements can shorten your descriptions with short notations. Such notations are ubiquitous in academic writings, so it is important to get them right!
 	
 	Loosely speaking, a statement is an observation or deduction, e.g. $x > 5$ states that $x$ is larger than $5$.
 	
 	Some common operations on statements:
 	\begin{itemize}
 		\item $\lnot A$: Negates the statement $A$.
 		\item $A \lor B$: Either $A$ or $B$, e.g. $x > 10 \lor x < 0$ means $x$ can be \emph{either} greater than $10$ or less than $0$.
 		\item $A \land B$: Both $A$ and $B$, e.g. $x \in X \land x > 5$ means $x$ must be \emph{both} in set $X$ and greater than $5$.
 	\end{itemize}
 	
 	\subsection{Logical Deduction Notations}
 	
 	Arrows are commonly used in logical deductions; however, note that different arrows have \textbf{completely different} meanings, and misusing arrows can cause confusions for the reader. For example, $\to$ and $\implies$ are completely different, and should never be used interchangeably.
 	
 	\begin{defn}
 		Statement $A$ \emph{implies} statement $B$ means that if $A$ is true, then $B$ is true. Such a relation is denoted $A \implies B$.
  	\end{defn}
  	
  	For example:
  	\begin{center}
  		Thomas is at Oxford $\implies$ Thomas is not in China
  	\end{center}
  	
  	\begin{defn}
  		An \emph{if and only if} relation on statement $A$ and $B$ denotes that $A$ and $B$ should either both be true, or  both be false, as the state of either statement can guarantee the state of the other. Denoted as $A \iff B$, the "if and only if" relation can be relaxed into two relations:
  		\begin{itemize}
  			\item \textbf{Necessary}: $A$ is needed for $B$ to be true.
  			\item \textbf{Sufficient}: Knowing that $A$ is true is enough for $B$ to be true.
  		\end{itemize}
  		If both requirements above are satisfied, then $A \iff B$.
  	\end{defn}
  	
  	For example:
  	\begin{center}
  		S is the empty set $\iff$ $|S| = 0$
  	\end{center}
  	
  	An alternative way of defining $A \iff B$ is:
  	\begin{gather*}
  		(A \implies B) \land (B \implies A)
  	\end{gather*}
  	
  	\subsection{Quantifiers}
  	
  	Quantifiers are symbols that define a variable with certain given semantics, i.e. information regarding the value of the variable. This is useful when formulating a predicate or statement.
  	
  	\begin{itemize}
  		\item $\forall a \in A$ (Universal Quantifier): For all $a$ in set $A$, i.e. $a$ can be set to any value in $A$.
  		\item $\exists a \in A$ (Existential Quantifier): There exists an $a$ i $A$.
  	\end{itemize}
  	
  	For example, the statement "all integers have an additive inverse" can be rewritten with quantifiers (different writings may use different conventions for separators):
  	\begin{gather*}
  		\forall x \in \mathbb{Z},\ \exists y \in \mathbb{Z}.\ x + y = 0
  	\end{gather*}
 	
 	\section{Number Theory}
 	
 	\begin{itemize}
 		\item $a \mid b$: $a$ is a factor of $b$, i.e. $b$ has remainder $0$ when divided by $a$.
 		\item $a \nmid b$: $a$ is not a factor of $b$, i.e. $a$ does not divide $b$.
 		\item $a \bmod b$: The remainder of $a$ divided by $b$.
 		\item $a \equiv b \bmod n$: $a$ is congruent to $b$ modulo $n$, i.e. $n \mid (b - a)$.
 		\item $\mathbb{Z}/n\mathbb{Z}$: The cyclic group (see \nameref{sec:terms}) formed by numbers $\bras{0, 1, \dots, n - 1}$ under multiplication modulo $n$.
 		\item $\lfloor x \rfloor$: The floor of $x$, i.e. the largest integer less than or equal to $x$.
 		\item $\lceil x \rceil$: The ceiling of $x$, i.e. the smallest integer greater than or equal to $x$.
 		\item $\lfloor x \rceil$ (or $\lbrack x \rbrack$): The nearest integer to $x$.
 		\item $a \perp b$: $a$ and $b$ are coprimes, i.e. the greatest common factor of $a$ and $b$ is $1$
. 	\end{itemize}
 	
 	\section{Calculus}
 	
 	The symbols in calculus are standard to what we've covered in class. However, there are some alternative ways of writing derivatives.
 	
 	Consider function $f(x, y)$. Its partial derivative with respect to $x$ can be written as:
 	\begin{itemize}
 		\item $\frac{\partial f}{\partial x}(x, y)$
 		\item $f_x(x, y)$
 		\item $D_xf(x, y)$
 	\end{itemize}
 	
 	Similarly, the second order derivative of $f(x, y)$ with respect to $x$ and then $y$ can be written as:
 	\begin{itemize}
 		\item $\frac{\partial^2 f}{\partial x \partial y}(x, y)$
 		\item $f_{xy}(x, y)$
 		\item $D_{xy}f(x, y)$
 	\end{itemize}
 	
 	Note that according to Clairaut's theorem, $f_{xy} = f_{yx}$, so the order of the "with respect to" variables does not matter.
 	
 	\subsection{Vector Calculus}
 	
 	The gradient of a real-valued function $f(x_1, \dots, x_n)$, denoted $\nabla f$, is defined as the vector:
 	\begin{gather*}
 	\nabla f = 
	\begin{bmatrix}
 		\frac{\partial f}{\partial x_1} \\
 		\vdots \\
 		\frac{\partial f}{\partial x_n}
 	\end{bmatrix}
 	\end{gather*}
 	
 	The gradient marks the steepest direction of descent for function $f$.
 	
 	Note that the operator $\nabla$ is often considered in its vector form, $\nabla = \braks{\frac{\partial}{\partial x_1}, \dots, \frac{\partial}{\partial x_n}}$. This makes it flexible, and can also be used on a vector-valued function:
 	\begin{itemize}
 		\item $\nabla \cdot f$: The divergence of $f$.
 		\item $\nabla \times f$: The curl of $f$.
 	\end{itemize}
 	
 	\begin{defn}
 		The \emph{Jacobian matrix} of a vector-valued function $f : \bras{x_1, \dots, x_n} \to \bras{f_1, \dots, f_m}$ is an $m \times n$ matrix containing each component of the result taken derivative to each components in the parameter:
 		\begin{gather*}
 		J_f = 
 		\begin{bmatrix}
 			\frac{\partial f_1}{\partial x_1} & \dots & \frac{\partial f_1}{\partial x_n} \\
 			\vdots & \ddots & \vdots \\
 			\frac{\partial f_m}{\partial x_1} & \dots & \frac{\partial f_m}{\partial x_n}
 		\end{bmatrix}
 		\end{gather*}
 	\end{defn}
 	
 	Note that there is no established convention for the orientation of the Jacobian matrix; some writings may feature a transposed version of the above definition.
 	
 	\section{Terminologies}
 	\label{sec:terms}
 	
 	\section{Appendix: \LaTeX\ Notations}
 	
 	This section aims to cover the typesetting of common symbols in \LaTeX.	 This includes all symbols shown in this document, as well as ones we've covered in class.
 	
 	\subsection{Sets Symbols}
 	
 	\begin{multicols}{2}
 	\begin{itemize}
	 	\item $\varnothing$: \ltc{varnothing}
 		\item $\NN$: \ltc{mathbb\{N\}}
 		\item $\mathbb{Z}$: \ltc{mathbb\{Z\}}
 		\item $\QQ$: \ltc{mathbb\{Q\}}
 		\item $\RR$: \ltc{mathbb\{R\}}
 		\item $\RR_+$: \ltc{mathbb\{R\}\_+}
 		\item $\RR_{++}$: \ltc{mathbb\{R\}\_\{++\}}
 		\item $\CC$: \ltc{mathbb\{C\}}
 		\item $\RR^n$: \ltc{mathbb\{R\}\^{}n}
 		\item $\RR^{m \times n}$: \ltc{mathbb\{R\}\^{}\{m \textbackslash times n\}}
 		\item $\bras{x}$: \ltc{lbrack x \textbackslash rbrack}
 		\item $|A|$: \ltcc{\textbackslash lvert A \textbackslash rvert}
 		\item $\overline{A}$: \ltc{overline\{A\}}
 		\item $a \in A$: \ltcc{a \textbackslash in A}
 		\item $A \subset B$: \ltcc{A \textbackslash subset B}
 		\item $A \not\subset B$: \ltcc{A \textbackslash not\textbackslash subset B}
 		\item $A \subseteq B$: \ltcc{A \textbackslash subseteq B}
 		\item $A \not\subseteq B$: \ltcc{A \textbackslash not\textbackslash subseteq B}
 		\item $A \setminus B$: \ltcc{A \textbackslash setminus B}
 		\item $A \cap B$: \ltcc{A \textbackslash cap B}
 		\item $A \cup B$: \ltcc{A \textbackslash cup B}
 		\item $A \sqcup B$: \ltcc{A \textbackslash sqcup B}
 		\item $A \times B$: \ltcc{A \textbackslash times B}
 		\item $\bigcap^n_{i=1} a_i$: \ltc{bigcap\^{}n\_\{i=1\} a\_i}
 		\item $\bigcup^n_{i=1} a_i$: \ltc{bigcup\^{}n\_\{i=1\} a\_i}
 		\item $a \neq b$: \ltcc{a \textbackslash neq b}
 	\end{itemize}	
 	\end{multicols}

	\subsection{Functions}
	
	\begin{multicols}{2}
 	\begin{itemize}
	 	\item $A \to A$: \ltcc{A \textbackslash to A}
	 	\item $\text{id}_A$: \ltc{text\{id\}\_A}
 	\end{itemize}	
 	\end{multicols}
	
\end{document}
